\documentclass[12pt, oneside]{article}   	% use "amsart" instead of "article" for AMSLaTeX format
\usepackage{geometry}                		% See geometry.pdf to learn the layout options. There are lots.
\geometry{letterpaper}                   		% ... or a4paper or a5paper or ... 
%\geometry{landscape}                		% Activate for rotated page geometry
%\usepackage[parfill]{parskip}    		% Activate to begin paragraphs with an empty line rather than an indent
\usepackage{graphicx}				% Use pdf, png, jpg, or eps§ with pdflatex; use eps in DVI mode
								% TeX will automatically convert eps --> pdf in pdflatex		
\usepackage{amssymb}

%SetFonts

%SetFonts


\title{ME 459 Final Project Proposal - Standing Balance Model Optimization}
\author{Jackson Fox}
%\date{}							% Activate to display a given date or no date

\begin{document}
\maketitle
\section{Project Overview}
\subsection{Problem Description}
Standing balance is an inherently unstable task that requires control of rotational and translation movement.  Humans use the interaction force between the ground and the foot to help manage the fluctuating accelerations of the body that can and do occur in standing balance.  The commonly used model for how the legs and feet interact with the ground, and in turn this interaction force, is a double inverted pendulum with torque actuated knee and ankle joints (see Figure 1).


In order to help elucidate the control strategy used during standing balance, this model can be simulated using a forward dynamics approach across a range of joint and limb positions to help identify 

\subsection{Motivation}
My research is with the Neuromuscular Coordination Laboratory under Dr. Kreg Gruben.  Our work focuses on understanding the methods by which humans exercise motor control, with a current specific focus on standing balance.  Dr. Gruben's research has identified specific intersection points in the location and angle of the interaction force with the ground through human balance studies, and it may be significant to the model that we use to constrain balance.  I want to use this project as an opportunity to more comprehensively identify the range of physically possible positions in standing balance using the double inverted pendulum model that satisfy standard constraints, so that we can look deeper into what these optimal positions are and how they overlap with the intersection point that we have seen in experimental data.
\subsection{Project Goals}

\section{Project Logistics}

\subsection{}

\end{document}  